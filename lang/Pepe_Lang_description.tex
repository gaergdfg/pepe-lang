\documentclass[12pt]{article}
\usepackage[utf8]{inputenc}
\usepackage[T1]{fontenc}
\usepackage[left=2cm, right=2cm, top=2cm]{geometry}
\usepackage{amsmath}


\begin{document}


Piotr Prabucki, 418377


\vspace{0.2cm}
\begin{center}
\vspace{1mm}
 \Large \textbf{Pepe lang}
\vspace{0.7cm}
\end{center}


\noindent
Pepe lang jest nieco rozszerzoną wersją języka Latte.

\begin{itemize}
    \item Trzy typy wartości: $int$, $bool$ i $string$.
    \item Literały, arytmetyka, porównania oraz zmienne i operacja przypisania jak w Latte.
    \item Jawne wypisywanie wartości na wyjście za pomocą funkcji $printInt$, $printBool$ i $printString$ dla odpowiednich typów.
    \item Bloki instrukcji konsturkcji $while$, $if$ z (i bez) $else$ muszą być otoczone nawiasami
        $$ while \; \text{( warunek )} \; \text{\{ instr \}} $$
        $$ if \; \text{( warunek )} \; \text{\{ instr \}} $$
        $$ if \; \text{( warunek )} \; \text{\{ instr \}} \; else \; \text{\{ instr \}}$$
    \item Operacje przerywające pętlę $while$ -- $break$ i $continue$.
    \item Funkcje przyjmujące i zwracające wartość dowolnych obsługiwanych typów, dostępna rekurencja oraz zagnieżdżanie funkcji.
    \item Przekazywanie parametrów przez zmienną oraz przez wartość.
    \item Obsługa błędów wykonania.
    \item Statyczne typowanie.
\end{itemize}


\end{document}
